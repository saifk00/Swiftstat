\section{DDG Trees} \label{sec:ddg-trees}

\subsection{Tree Preliminaries}

\subsection{DDG Tree Introduction}

Discrete distribution generating trees (DDG trees), first introduced in \cite{ALGCPLX:KnuYao76}, are a form of binary tree that permit efficient and exact sampling from discrete probability distributions.
At a high level, DDG trees consist of two types of nodes: branch nodes (all non-leaf nodes) and terminal nodes (leaf nodes associated with a value). 
Sampling consists of traversal down the tree, with a random coin flipped at each branch node to determine whether to traverse down the left or right child until a terminal node is hit.
This terminal node's associated value is the sample.
If a DDG tree is properly constructed, then the probability of a possible sample output can match that of a desired distribution. 
The key insight to achieve this is to admit terminal nodes at any level of the tree, and to allow multiple terminal nodes to be associated with the same value.
A useful concrete example is given in Figure~\ref{fig:todo}.

While DDGtrees may be finite or infinite, only finite trees are in the scope of our work.
Concretely, this means we only consider DDG trees that represent discrete probability distributions where the probabilities can be described as finite binary strings. 
We now formally define DDG trees.

\begin{definition}[DDG Tree (adapted from \cite{Devroye86})]
Let $D:\mathbb{V} \to \{0, 1\}^n$ be a discrete distribution with binary probabilities. Then a \emph{DDG tree matching $D$} is a finite binary tree $T$ consisting of the following two types of nodes.
\begin{itemize}
	\item Branch nodes, which are unmarked and must have exactly two children, each of which can be a branch node or terminal node, and
	\item Terminal nodes, which are marked with some value $v \in \mathbb{V}$ and may not have any children.
\end{itemize}
$T$ must also satisfy an additional constraint that ensures sampling reflects the distribution $D$. In particular, for all $v \in \mathbb{V}$, let $T_v = \{ x_v \in \textsc{TerminalNodes}(T) \mid x_v.\mathsf{value} = v\}$ be the set of terminal nodes in $T$ marked with value $v$. Then
$$ \sum_{x_v \in T_v} 2^{-\textsc{Level}(T, x_v)} = D(v).$$
\end{definition}

Observe that, for a given discrete probability distribution $D$, there are any number of DDG trees that match $D$.
Figure~\ref{fig:todo} provides three concrete examples of DDG trees that match the same distribution.
We now consider two additional definitions that clarify this sundry of trees.

\begin{definition}[Minimal DDG Tree]
Let $D:\mathbb{V} \to \{0, 1\}^n$ be a discrete distribution with binary probabilities. A DDG tree $T$ matching $D$ is \emph{minimal} if there does not exist any DDG tree $T'$ matching $D$ such that $\textsc{Depth}(T') < \textsc{Depth}(T)$.
\end{definition}

\begin{definition}[Canonical DDG Tree]
Let $D:\mathbb{V} \to \{0, 1\}^n$ be a discrete distribution with binary probabilities. A DDG tree $T$ matching $D$ is \emph{left-sided canonical} if, for all terminal nodes $x \in \textsc{TerminalNodes}(T)$, any one of the following conditions holds.
\begin{itemize}
	\item $T$ only contains $x$ (i.e. $x = T.\mathsf{root}$ so $x$ is parentless).
	\item If $x$ is a left child (i.e. $x = x.\mathsf{parent}.\mathsf{left}$), then TODO FIND BETTER DEFINITION.
\end{itemize}
\end{definition}
